\documentclass[a4paper,10pt]{article}
\usepackage{CJK}
\usepackage{indentfirst}
\usepackage{graphicx}
\setlength{\parindent}{2em}
\setlength{\parskip}{0.2em}
\setlength{\itemsep}{0.1ex}

\begin{document}
\begin{CJK}{UTF8}{gbsn}
\tableofcontents
\newpage
\section{数据类型}
\subsection{对象}
Codger语言的整体的设计思想为:一切皆是对象。整数,浮点数,字符串等基本的数据类型是对象,函数是对象,甚至类也是对象\ldots{} 因为这种特性,可以在源程序中使用类似于下面的语句:
\begin{quote}
\begin{verbatim}
print 1.add(3)  #输出1+3的值
print "Codger".length() #输出字符串"Codger"的长度
\end{verbatim}
\end{quote}
对象Object是所有对象的父类,也就是说所有对象直接或者间接的继承对象Object。每个对象所携带的信息分为两部分:1)对象的类型信息,用于标识该对象的类型,所有同类型的对象共享对象的类型信息,类型信息里面保存了对象的所属类型,对象使用私有数据的方法,对象与其它对象交互的方式等。类型信息在对象创建时绑定到对象上。2)对象的私有数据,不同的对象有不同的私有数据,不同类型的对象对私有数据的使用方法也不同,例如整数对象的私有数据部分用于保存数值,字符串对象的私有数据用于保存字符序列。
\subsection{变量}
Codger是一门动态类型语言,所有对象的类型都是在运行时才能确定,变量在使用时不需要预先申明,也不需要申明变量的类型,可以不同的时刻给变量赋值不同类型的对象,例如下面的语句:
\begin{quote}
\begin{verbatim}
a=2          #把整数对象2赋值给变量a
a="Codger"   #把字符串对象"Codger"赋值变量a 
\end{verbatim}
\end{quote}
当对一个变量赋值时,变量中会保存到该对象的引用,如果访问一个未引用对象的变量时,虚拟机会抛出变量未定义的异常。当把一个变量赋值给另一个变量,两个变量所引用的对象为同一个,例如:
\begin{quote}
\begin{verbatim}
a=[3,4]       #变量a引用一个数组对象
b=a           #把变量a赋值给变量b
a.push(5)     #调用a引用数组对象的push方法
print b       #输出b,结果为 [3,4,5]
\end{verbatim}
\end{quote}
\subsection{基本数据类型}
codegr现在支持8种基本数据类型:布尔值,整数,长整数,浮点数,字符串,数组,散列表,以及特殊类型Nil用于表示空对象,它们都可以在程序中显示定义。布尔值,整数,长整数,以及浮点数四种类型属于标量类型,一旦标量类型的对象被创建,在其生命周期内,对象的状态不会发生改变,其中:
\begin{enumerate}
\item 布尔值用于逻辑运算,也被称为逻辑值,它的值有两个,分别为真和假,假用`false'表示,真用`true'表示;
\item 整数,浮点数,长整数,用于数值运算\footnote{它们3个以及字符串的定义规则见词法部分};
\item 字符串是由多个字符组成的有限序列;
\end{enumerate}

另一种为可变类型,可变类型的对象被创建后,状态可能发生多次改变,数组和散列表属于可变类型。其中:
\begin{enumerate}
\item 数组是一种顺序性的容器,用于存储对象,定义规则为:
\begin{quote}
\begin{verbatim}
"[" [expr] { "," expr} [","] "]"
\end{verbatim}
\end{quote}
\item 散列表则是一种关联性容器,散列表中每个元素都有键和值组成,键用于索引,值用于保存数据,定义规则为:
\begin{quote}
\begin{verbatim}
"{" [ expr "->" expr ] { "," expr "->" expr } [","] "}"
\end{verbatim}
\end{quote}
\end{enumerate}
定义规则使用EBNF文法表示\footnote{EBNF的全称为:Extended Backus-Naur Form,后面涉及到的文法全部都为EBNF文法},expr表示表达式,双引号中表示字符序列, [\ldots{}]表示可选,\{\ldots{}\}表示重复一次或多次。

特殊类型Nil用于表示空对象,类似于java中的NULL对象或者是python中的None对象,在源程序中用`Nil'来访问。

\subsection{函数与匿名函数}
在Codger中一切皆为对象,函数也不例外。函数可以赋值给一个变量或者是保存在容器中,在解析器解析源程序,函数对象并不会被创建,这一点可能和其它动态语言差别很大,只有虚拟机运行到定义函数的字节码时,函数对象才被创建,同时把函数对象保存在定义函数名称同名的变量中,函数对象创建后,可以通过变量访问函数对象。如果在函数定义时,函数的名称没有指定,则该函数被视为匿名函数,匿名函数和函数属于同一种对象类型。
\subsubsection{函数创建}
每当函数对象创建时,会对当前作用域引用,当前作用域退出时,但不会被立即销毁,只有所有引用该作用域的函数对象毁销时,该作用域才会被毁销。只要函数对象存活,函数对象所在的作用域中的所有变量都被保存了下来,形成词法闭包\footnote{词法闭包:引用了自由变量的函数。这个被引用的自由变量将和这个函数一同存在,即使已经离开了创造它的环境也不例外}。如果某个函数对象所在作用域只被该函数对象引用,那么该作用域就可被视为该函数对象的私用数据空间。
\subsubsection{函数调用}
当函数对象被调用时,会创建一个新的作用域,新作用域的上层作用域会被设置为函数对象所在的作用域。多次调用函数对象,会创建多个作用域,保证了函数调用时内部变量的独立性。在Codger中变量的访问只会在当前作用域用查找,并不会查找上层作用域。在变量前面加上\$表示查找上层到顶层作用域中的变量,变量的赋值也是一样。

如果一个函数对象直接或者间接调用自身,则被称递归调用,但对于动态语言来说,递归调用的代价很高,在递归过程中的临时变量会累积起来而得不到释放,如果递归深度过大,会消耗掉很大一块的内存,同时也加大了垃圾回收例程的负担。当递归算法可以使用等价的迭代算法实现时,优先选择迭代算法。
\subsubsection{函数定义}
在Codger中使用下面的语法定义函数:
\begin{quote}
\begin{verbatim}
"func" [identifier] "(" [arg] { "," arg } [","] ")"
        stmts
"end"
\end{verbatim}
\end{quote}
其中identifier为标识符,表示函数名,当函数名为空时,则为匿名函数。arg表示函数的参数类型,参数类型有3种:
\begin{description}
\item[简单参数:]只由参数名组成,当函数调用时,参数的个数比须与定义中简单参数的个数相匹配。
\item[缺省参数:]由参数名和表达式组成:\verb|identifier "=" expr|,缺省参数必须位于简单参数的后面,当函数调用时,参数个数比须大于等于简单参数的个数,参数的个数的最大值不能超过简单参数与缺省参数的总和,如果在调用时,缺省参数部分的值没有被指定,则会传入其相对应的\verb|expr|的值。例如:
\begin{quote}
\begin{verbatim}
func add_default(x,y=10)
    return x+y
end 
print add_default(1)     #输出结果为11
print add_default(1,20)  #输出结果为21
\end{verbatim}
\end{quote}
\item[多参数:]在参数名前面加上*,表示接受多个参数,多参数只能出现在参数定义的最后面,并且最多只能定义一个。当函数调用时,多参数会以数组的形式传入。例如:
\begin{quote}
\begin{verbatim}
func add_many(*x)
    sum=0
    for i in x            #遍边数组中的每一个元素
        sum+=i
    end
    return sum
end
print add_many()          #输出为0
print add_many(1,2)       #输出为3
print add_many(1,2,3,4)   #输出为10
\end{verbatim}
\end{quote}
\end{description}
如果在函数定义时,没有return语句,则返回对象Nil
\subsection{类与实例}
\subsection{模块} 








\end{CJK}
\end{document}