\addcontentsline{toc}{section}{\large引言}
\noindent{\Large引\quad言}
\newline
这篇论文主要介绍一门面向对象,支译式的动态程序设计语言Codger,以及用于运行Codger语言的解析器的内部原理,论文的内容有:
\begin{description}
\item[第一章]简单介绍了Codger语言。
\item[第二章]讲述了Codger语言单词是基本词法单位--单词,并且给出了构成每个单词的正则文法。
\item[第三章]介绍了在Codger开发过程中,总结出的一种高效,简单,而且易于手工构造的词法识别算法。
\item[每四章]介绍了在编写Codger语言时可以使用了基本数据类型,每种数据类型的使用方法以及它些数据类型的特性。这部分中还介绍了怎么在Codger中使用类,函数,以及用于管理和抽象源文件的模块对象。
\item[第五章]介绍了Codger语言中的运算符与表达式,以及这些运算符之间的代先级关系。
\item[第六章]介绍了使用Codger编写程序的基本语法,并且给出了每种类型语句相应于EBNF。这部分还为每种类型的语句,给出了一个小的,示例性的代码。
\item[第七章]给出了每种不同类型的语句转换成字节码的框架。这些转换框架都在Codger解析器中实际开发中得到了验证。
\item[第八章]介绍了Codger虚拟机的组成,以及每个模块的作用。
\item[第九章]Codger解析器中的内存管理系统,这部分给出了一个小内存的快速分配算法,它比c库能快到\verb|3~4|倍左右;Codger语言中的垃圾回收系统所使用的分代式节点复制法,以及怎么在页的基础上构造不连续性的堆。
\end{description}

