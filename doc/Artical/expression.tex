\section{运算符与表达式}
表达式\footnote{表达式的完整EBNF文法请参考附录2}是由运算符将运算对象连接起来具有合法语义的式子,运算符分为这么三类:一元运算符,后缀运算符,二元运算符。不同类型的对象对于运算符的处理方式不同,对于运算符的处理方法保存在对象的类型信息里面,例如当二元运算符+的左右操作数都为字符串对象时,返回结果为连接后的字符串;当操作数为左右整数对象时,返回结果为两个数的和。

\subsection{一元运算符}
一元运算符也称为前缀运算符,有这么4个:\begin{textbf}+, -, \verb|~|, not\end{textbf}
\begin{enumerate}
\item 运算符\textbf{ not} 用于逻辑运算,表示逻辑非;
\item 运算符\textbf{ +} 在数值运算中,表示负号。
\item 运算符\textbf{ -} 在数值运算中,表示正号。
\item 运算符\verb| ~| 在数值运算中,表示按位取反
\end{enumerate}
\subsection{后缀运算符}
在Codger后缀运算符有\textbf{ [ ] , . , ( ) } 这么3个
\begin{enumerate}
\item 运算符\textbf{[ ]},也被称为下标运算符的使用方法为:
\begin{quote}
\begin{verbatim}
expr1 "[" expr2 "]"
\end{verbatim}
\end{quote}
\begin{itemize}
\item 当expr1为字符串对象时,表示下标访问,返回结果为一个新的字符串,该字符串的内容为位于 expr1 对象的expr2位置的字符。
\item 当expr1为数组对象时,表示下标访问,返回结果为存储在 expr1 对象的expr2位置的对象。
\item 当expr1为散列表对象时,表示查询,返回结果为存储在expr1对象中健值expr2所关联的对象。
\end{itemize}
\item 运算符 \textbf{ . } 用于访问对象的属性,所以也被称为属性访问运算符,使用方法为:
\begin{quote}
\begin{verbatim}
expr "." identifier 
\end{verbatim}
\end{quote}
其中identifier为标识符,表示访问对象expr的identifier属性。
\item 运算符 \textbf{ ( ) } 用于过程调用,使用方法为:
\begin{quote}
\begin{verbatim}
expr "(" [ expr ] {"," expr} [","] ")"
\end{verbatim}
\end{quote}
当expr为函数对象时,则发生过程调用;当expr为类对象时,则会创建一个实例对象。
\end{enumerate}
\subsection{二元运算符}
 Codger中的二元运算符有18个,分别为:
\begin{quote}
\begin{verbatim}
+  -  *  /  %  
<<  >>  &  ^  |  
<  <=  ==  !=  >=  > 
and  or 
\end{verbatim}
\end{quote}
这18个运算符又被分为4类:
\begin{enumerate}
\item 算术运算符:
\begin{quote}
\begin{verbatim}
+  -  *  /  %  
\end{verbatim}
\end{quote}
在数值运算中,运算符 + 用于求和;,- 用于求差,* 用于求乘积; / 用于求商; \% 用于取模。
\item 位运算符:
\begin{quote}
\begin{verbatim}
<<  >>  &  ^  |  
\end{verbatim}
\end{quote}
在数值运算中,运算符 \verb << {} 表示左移运算,\verb >> {} 表示右移运算, \verb & {} 表示按位与运算, \verb ^ {} 表示按位异或运算,\verb | {} 表示按位或运算
\item 比较运算符:
\begin{quote}
\begin{verbatim}
<  <=  ==  !=  >=  > 
\end{verbatim}
\end{quote}
在数值运算中,运算符 \verb < {} 表示小于运算,\verb <= {} 表示小于或等于运算, \verb == {} 表示等于运算,\verb != {} 表示不等于运算,\verb >= {} 表示大于或等于运算, \verb > {} 表示大于运算。
\item 逻辑运算符:
\begin{quote}
\begin{verbatim}
and  or 
\end{verbatim}
\end{quote}
运算符and表示逻辑与运算,如果左边表达式返回的左操作数为假,则立刻返回假,而不会对右边表达进行运算;运算符or表示逻辑或运算,如果左边表达式返回的结果右操作数为真,则立刻返回真,而不会对右边表达式进行运算。逻辑运算中,当操作数不为布尔值,会进行布尔转换:整数,浮点数,长整数为0时为假,其它情况为真;数组与散列表中的元素个数为0时,为假,其它情况为真;字符串为空串时为假,其它情况为真;特殊类型Nil为假
\end{enumerate}

\subsection{类型转换}
在数值运算中,二元运算符的左右操作数为不同的数据类型时,会涉及到类型的转换,转换的优先级为:浮点数 \verb > {} 长整数 \verb > {} 整数 \verb > {} 布尔值,当转换优先级高的数据类型与转换优先级低的数据类型运算时,转换优先级低的数据类型会转换到优先级高的数据类型。

\subsection{运算符优先级}
表达式由多个运算符组成,合个运算之间的优先级别关系以及同级别运算符的运算顺序(结合方向)如下表: 
\begin{center}
\begin{tabular}{c|c|c|c}
优先级别&运算符&运算形式&结合方向 \\
\hline
&( )&(e)& \\
1&[ ]&a[e]& 从左至右 \\
& . & a.x& \\
\hline
2&+ - \verb|~| &+a&从右至左 \\
\hline
3&* / \% &e1*e2&从左至右 \\
\hline
4& + - &e1-e2&从左至右 \\
\hline 
5& \verb|<< >>| &e1 \verb|<<| e2&从左至右 \\ 
\hline
6& \verb|&| &e1 \verb|&| e2&从左至右 \\
\hline
7& \verb|^| &e1 \verb|^| e2 &从左至右 \\
\hline
8& \verb | &e1 \verb |  e2 &从左至右  \\
\hline 
9& \verb|<  <=  >  >=|& e1 \verb|<| e2 &从左至右 \\
\hline
10& \verb|==  !=| & e1 \verb|==| e2 &从左至右 \\
\hline 
11& not & not e1 & 从左至右 \\
\hline 
12& and & e1 and e2 & 从左至右 \\
\hline
13& or & e1 or e2 & 从左至右 \\
\hline
\end{tabular}
\end{center}
在表中,优先级的数值越大,其优先级越低。优先级最高的为括号,最低的为运算符 or。