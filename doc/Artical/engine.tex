\documentclass[a4paper,10pt]{article}
\usepackage{CJK}
\usepackage{indentfirst}
\usepackage{graphicx}
\setlength{\parindent}{2em}
\setlength{\parskip}{0.2em}
\setlength{\itemsep}{0.1ex}
\begin{document}
\begin{CJK}{UTF8}{gbsn}
\section{虚拟机}
Codger解析器整体大的框架可以分为两部分:解析器前端后解析器后端。解析器前端负责检查源程序中词法、语法是否正确,并且把源程序转换成与之对应的字节码。解析器后端负责运行字节码的运行,模块管理,栈帧管理,异常处理,内存管理,数据栈管理,解析器后端也被称为虚拟机。
\subsection{虚拟机的组成}
在Codger中每个虚拟机实例用EgThread来表示,EgThread可以创建多个,现在Codger还不支持并发运行多个EgThread,但可以模拟并发,让每个虚拟机实例轮流的以串行的方式执行,一个虚拟机实例执行一定数量的字节码后,然后调度另一个虚拟机实例。

每当一个虚拟机实例EgThread被创建时,它将会得到自己的pc寄存器,sp 寄存器,一个数据栈,以及一个Codger栈。寄存器pc的值总是会指向下一条执行的字节码;数据栈用于保存在计算过程中的临时对象;寄存器sp总是指向数据栈的栈顶;Codger栈总是存储Codger方法的调用状态,包括它被调用时传进来的参数值,它的返回值,局部变量等。

Codger方法指该方法在Codger源程序定义方法,如果一个方法调用是用c语言实现的,那么该方法被称为本地方法。Codger是由许多栈帧组成,一个栈帧包括一个Codger方法调用状态,当虚拟机调用一个Codger方法时,会将一个新的帧栈压入到Codger栈中,当方法返回时,这个栈帧会作Codger栈中弹出并抛弃。
\subsection{模块对象}
\subsection{数据栈}
\subsection{栈帧}
\subsection{异常处理}
\subsection{字节码}
\section{内存管理}
\subsection{内存分配算法}
\subsection{垃圾回收算法}

\end{CJK}
\end{document}
