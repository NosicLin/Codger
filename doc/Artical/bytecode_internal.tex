\section{字节码的内部执行过程}
\subsection{不带参数的字节码}
\begin{enumerate}
\item OP\_CALL: 方法调用,需要操作数2个,例如: cos(3) \\
执行过程为:
\begin{quote}
\begin{verbatim}
r1=dstack.pop()
r0=dstack.pop()
acc=r0.call(Nil,r1)
dstack.push(acc)
\end{verbatim}
\end{quote}

\item OP\_CALL\_WITH\_HOST: 调用对象的方法,需要操作数3个,例如:a.add(3)\\
执行过程为:
\begin{quote}
\begin{verbatim}
r2=dstack.pop()
r1=dstack.pop()
r0=dstack.pop()
acc=r1.op_call(r1,r2)
dstack.push(acc)
\end{verbatim}
\end{quote}

\item OP\_SET\_ITEM: 给集合指定的元素赋值,需要3个操用数。例如: a[3]=4  \\
执行过程为: 
\begin{quote}
\begin{verbatim}
r2=dstack.pop()
r1=dstack.pop()
r0=dstack.pop()
r1.set_item(r2,r0)
\end{verbatim}
\end{quote}

\item OP\_GET\_ITEM: 访问集合中指定的元素,需要2个操作数,例如: a[3] \\
执行过程为:
\begin{quote}
\begin{verbatim}
r1=dstack.pop()
r0=dstack.pop()
acc=r0.get_item(r1)
dstack.push(acc)
\end{verbatim}
\end{quote}

\item OP\_POSITIVE: 一元运算符 + ,需要1个操作数,例如: +a \\
执行过程为:
\begin{quote}
\begin{verbatim}
r0=dstack.pop()
acc=r0.positive()
dstack.push(acc) 
\end{verbatim}
\end{quote}

\item OP\_NEGATIVE: 一元运算符 - ,需要1个操作数,例如:-a \\
执行过程为:
\begin{quote}
\begin{verbatim}
r0=dstack.pop()
acc=r0.negative()
dstack.push(acc)
\end{verbatim}
\end{quote}

\item OP\_NEGATED: 一元运算符 \verb|~| ,需要1个操作数,例如:\verb|~|a \\
执行过程为:
\begin{quote}
\begin{verbatim}
r0=dstack.pop()
acc=r0.negated()
dstack.push(acc)
\end{verbatim}
\end{quote}

\item OP\_MUL: 二元运算符 * ,需要2个操作数,例如:1*2 \\
执行过程为:
\begin{quote}
\begin{verbatim}
r1=dstack.pop()
r0=dstack.pop()
acc=r0.mul(r1)
dstack.push(acc)
\end{verbatim}
\end{quote}

\item OP\_DIV:二元运算符 \verb|/| ,需要2个操作数,例如:1\verb|/|2 \\
执行过程为:
\begin{quote}
\begin{verbatim}
r1=dstack.pop()
r0=dstack.pop()
acc=r0.div(r1)
dstack.push(acc)
\end{verbatim}
\end{quote}

\item OP\_MOD:二元运算符 \verb|%| ,需要2个操作数,例如:1\verb|%|2 \\
执行过程为:
\begin{quote}
\begin{verbatim}
r1=dstack.pop()
r0=dstack.pop()
acc=r0.mod(r1)
dstack.push(acc)
\end{verbatim}
\end{quote}

\item OP\_PLUS:二元运算符 \verb|+| ,需要2个操作数,例如:1\verb|+|2 \\
执行过程为:
\begin{quote}
\begin{verbatim}
r1=dstack.pop()
r0=dstack.pop()
acc=r0.plus(r1)
dstack.push(acc)
\end{verbatim}
\end{quote}

\item OP\_MINUS:二元运算符 \verb|-| ,需要2个操作数,例如:1\verb|-|2 \\
执行过程为:
\begin{quote}
\begin{verbatim}
r1=dstack.pop()
r0=dstack.pop()
acc=r0.minus(r1)
dstack.push(acc)
\end{verbatim}
\end{quote}

\item OP\_LSHIFT:二元运算符 \verb|<<| ,需要2个操作数,例如:1\verb|<<|2 \\
执行过程为:
\begin{quote}
\begin{verbatim}
r1=dstack.pop()
r0=dstack.pop()
acc=r0.lshift(r1)
dstack.push(acc)
\end{verbatim}
\end{quote}

\item OP\_RSHIFT:二元运算符 \verb|>>| ,需要2个操作数,例如:1\verb|>>|2 \\
执行过程为:
\begin{quote}
\begin{verbatim}
r1=dstack.pop()
r0=dstack.pop()
acc=r0.rshift(r1)
dstack.push(acc)
\end{verbatim}
\end{quote}

\item OP\_AND:二元运算符 \verb|&| ,需要2个操作数,例如:1\verb|&|2 \\
执行过程为:
\begin{quote}
\begin{verbatim}
r1=dstack.pop()
r0=dstack.pop()
acc=r0.and(r1)
dstack.push(acc)
\end{verbatim}
\end{quote}

\item OP\_XOR:二元运算符 \verb|^| ,需要2个操作数,例如:1\verb|^|2 \\
执行过程为:
\begin{quote}
\begin{verbatim}
r1=dstack.pop()
r0=dstack.pop()
acc=r0.xor(r1)
dstack.push(acc)
\end{verbatim}
\end{quote}

\item OP\_OR:二元运算符 \verb^|^ ,需要2个操作数,例如:1\verb^|^2 \\
执行过程为:
\begin{quote}
\begin{verbatim}
r1=dstack.pop()
r0=dstack.pop()
acc=r0.or(r1)
dstack.push(acc)
\end{verbatim}
\end{quote}

\item OP\_LT:二元运算符 \verb|<| ,需要2个操作数,例如:1\verb|<|2 \\
执行过程为:
\begin{quote}
\begin{verbatim}
r1=dstack.pop()
r0=dstack.pop()
acc=r0.lt(r1)?true:false
dstack.push(acc)
\end{verbatim}
\end{quote}

\item OP\_LE:二元运算符 \verb|<=| ,需要2个操作数,例如:1\verb|<=|2 \\
执行过程为:
\begin{quote}
\begin{verbatim}
r1=dstack.pop()
r0=dstack.pop()
acc=r0.le(r1)?true:false
dstack.push(acc)
\end{verbatim}
\end{quote}

\item OP\_EQ:二元运算符 \verb|==| ,需要2个操作数,例如:1\verb|==|2 \\
执行过程为:
\begin{quote}
\begin{verbatim}
r1=dstack.pop()
r0=dstack.pop()
acc=r0.eq(r1)?true:false
dstack.push(acc)
\end{verbatim}
\end{quote}

\item OP\_NE:二元运算符 \verb|!=| ,需要2个操作数,例如:1\verb|!=|2 \\
执行过程为:
\begin{quote}
\begin{verbatim}
r1=dstack.pop()
r0=dstack.pop()
acc=r0.ne(r1)?true:false
dstack.push(acc)
\end{verbatim}
\end{quote}

\item OP\_GE:二元运算符 \verb|>=| ,需要2个操作数,例如:1\verb|>=|2 \\
执行过程为:
\begin{quote}
\begin{verbatim}
r1=dstack.pop()
r0=dstack.pop()
acc=r0.ge(r1)?true:false
dstack.push(acc)
\end{verbatim}
\end{quote}

\item OP\_GT:二元运算符 \verb|>| ,需要2个操作数,例如:1\verb|>|2 \\
执行过程为:
\begin{quote}
\begin{verbatim}
r1=dstack.pop()
r0=dstack.pop()
acc=r0.gt(r1)?true:false
dstack.push(acc)
\end{verbatim}
\end{quote}

\item OP\_BOOL:布尔类型转换,需要1个操作数\\
执行过程为:
\begin{quote}
\begin{verbatim}
r0=dstack.pop()
acc=r0.bool()
dstack.push(acc)
\end{verbatim}
\end{quote}

\item OP\_BOOL\_NOTAKE:布尔类型转换,需要1个操作数,但操作数不会被弹出数据栈\\
执行过程为:
\begin{quote}
\begin{verbatim}
r0=dstack.top()
acc=r0.bool()
dstack.push(acc)
\end{verbatim}
\end{quote}

\item OP\_LOGIC\_NOT:逻辑运算符not,需要1个操作数,例如:not 1 \\
执行过程为:
\begin{quote}
\begin{verbatim}
r0=dstack.pop()
acc=not r0.bool()
dstack.push(acc)
\end{verbatim}
\end{quote}

\item OP\_PRINT:输出指令,需要1个操作数,例如: print ``Codger'' \\
执行过程为:
\begin{quote}
\begin{verbatim}
r0=dstack.pop()
r0.print()
\end{verbatim}
\end{quote}

\item OP\_PRINT\_LN:输出换行符指令,不需操作数,例如:print \\
执行过程为:
\begin{quote}
\begin{verbatim}
print("\n")
\end{verbatim}
\end{quote}

\item OP\_ITER:获取对象的迭代器,需要1个操作数 \\
执行过程为:
\begin{quote}
\begin{verbatim}
r0=dstack.pop()
acc=r0.iter()
dstack.push(r0)
\end{verbatim}
\end{quote}

\item OP\_ITER\_NEXT:获取迭代器的下一个元素,需要1个操作数,但操作数不会被弹出数据栈 \\
执行过程为:
\begin{quote}
\begin{verbatim}
r0=dstack.top()
acc=r0.iter_next()
if acc != iter_stop
    pc+=3
    dstack.push(acc)
end
\end{verbatim}
\end{quote}

\item OP\_ARRAY\_BEGIN:创建一个数组对象,需要0个操作数 \\
执行过程为:
\begin{quote}
\begin{verbatim}
acc=Array()       #创建函数对象
dstack.push(acc)
\end{verbatim}
\end{quote}

\item OP\_ARRAY\_PUSH:向数组对象的尾部增加元素,需要2个操作数 \\
执行过程为:
\begin{quote}
\begin{verbatim}
r0=dstack.pop()
r1=dstack.pop()
r0.push_back(r1)
dstack.push(r0)
\end{verbatim}
\end{quote}

\item OP\_ARRAY\_END:保留指令
\item OP\_FUNC\_BEGIN:创建一个函数对象,需要0个操作数 \\
执行过程为:
\begin{quote}
\begin{verbatim}
acc=Func()                 #创建函数对象
acc.set_scope(cur_scope)   #引用当前作用域
dstack.push(acc)
\end{verbatim}
\end{quote}

\item OP\_FUNC\_DEFAULT\_ARGS:设置函数对象的默认参数的值,如果函数有多个默认参数,该指令会把值赋给第一个还未设置的默认参数,需要2个操作数 \\
执行过程为:
\begin{quote}
\begin{verbatim}
r1=dstack.pop()
r0=dstack.pop()
r0.set_defalut_arg(r1)
dstack.push(r0)
\end{verbatim}
\end{quote}

\item OP\_CLASS\_INHERIT:类对象继承另一个类对象,需要操作数2个 \\
执行过程为:
\begin{quote}
\begin{verbatim}
r1=dstack.pop()
r0=dstack.pop()
r0.set_inherit(r1)
dstack.push(r0)
\end{verbatim}
\end{quote}

\item OP\_LOAD\_NIL:把空对象Nil压入数据栈,需要0个操作数  \\
执行过程为:
\begin{quote}
\begin{verbatim}
acc=Nil
dstack.push(acc)
\end{verbatim}
\end{quote}

\item OP\_DISCARD:从数据栈中弹出一个对象,并抛弃,需要0个操作数 \\
执行过程为:
\begin{quote}
\begin{verbatim}
dstack.pop()
\end{verbatim}
\end{quote}

\item OP\_DUP\_DATA1:复制数据栈栈顶的对象,需要0个操作数 \\
执行过程为:
\begin{quote}
\begin{verbatim}
r0=dstack.top()
dstack.push(r0)
\end{verbatim}
\end{quote}

\item OP\_DUP\_DATA2:复制数据栈栈顶的2个对象,需要0个操作数 \\
执行过程为:
\begin{quote}
\begin{verbatim}
r2=dstack[sp-1]
r1=dstack[sp-2]
dstack.push(r0)
dstack.push(r1)
\end{verbatim}
\end{quote}

\item OP\_DUP\_DATA3:复制数据栈栈顶的3个对象,需要0个操作数 \\
执行过程为:
\begin{quote}
\begin{verbatim}
r2=dstack[sp-1]
r1=dstack[sp-2]
r0=dstack[sp-3]
dstack.push(r0)
dstack.push(r1)
dstack.push(r2)
\end{verbatim}
\end{quote}

\item OP\_DATA\_SWAP0\_1:交换数据栈栈顶第0个对象与第1个对象,需要0个操作数 \\
执行过程为:
\begin{quote}
\begin{verbatim}
acc=dstack[sp-1]
dstack[sp-1]=dstack[sp-2]
dstack[sp-2]=acc
\end{verbatim}
\end{quote}

\item OP\_DATA\_SWAP0\_2:交换数据栈栈顶第0个对象与第2个对象,需要0个操作数 \\
执行过程为:
\begin{quote}
\begin{verbatim}
acc=dstack[sp-1]
dstack[sp-1]=dstack[sp-3]
dstack[sp-3]=acc
\end{verbatim}
\end{quote}

\item OP\_DATA\_SWAP0\_3:交换数据栈栈顶第0个对象与第3个对象,需要0个操作数 \\
执行过程为:
\begin{quote}
\begin{verbatim}
acc=dstack[sp-1]
dstack[sp-1]=dstack[sp-4]
dstack[sp-4]=acc
\end{verbatim}
\end{quote}

\item OP\_EXIT:退出指令,需要0个操作数 \\
执行过程为:
\begin{quote}
\begin{verbatim}
EgThread.exit()
\end{verbatim}
\end{quote}

\item OP\_RETURN:栈帧返回指令,需要1个操作数 \\
执行过程为:
\begin{quote}
\begin{verbatim}
r0=dstack.pop()
EgThread.return(r0)
\end{verbatim}
\end{quote}

\item OP\_RETURN\_NIL:栈帧返回指令,需要0个操作数 \\
执行过程为:
\begin{quote}
\begin{verbatim}
r0=Nil
EgThread.return(r0)
\end{verbatim}
\end{quote}
\end{enumerate}




\subsection{带参数的字节码}
\begin{enumerate}
\item OP\_GET\_ATTR:访问对象成员,需要1个操作数 \\
执行过程为:
\begin{quote}
\begin{verbatim}
r0=dstack.pop()
r1=symbols_pool[rd]
acc=r0.get_attr(r1,r0==sframe.host)
dstack.push(acc)
\end{verbatim}
\end{quote}
其中sframe表示当前的栈帧,host为sframe的属主对象,rd是指令的参数。

\item OP\_SET\_ATTR:给对象的成员赋值,需要2个操作数 \\
执行过程为:
\begin{quote}
\begin{verbatim}
r0=dstack.pop()
r1=dstack.pop()
r2=symbols_pool[rd]
r1.set_attr(r2,r0,r1==sframe.host)
\end{verbatim}
\end{quote}

\item OP\_CLASS\_BEGIN:创建一个类对象,需要0个操作数 \\
执行过程为:
\begin{quote}
\begin{verbatim}
acc=Class()
r0=symbols_pool[rd]
acc.set_name(r0)
dstack.push(acc)
\end{verbatim}
\end{quote}

\item OP\_CLASS\_TEMPLATE:向类的模板中增加元素,需要2个操作数 \\
执行过程为:
\begin{quote}
\begin{verbatim}
r2=perm_symbols_pool[rd]
r1=dstack.pop()
r0=dstack.pop()
r0.add_template(r2,r1)
dstack.push(r0)
\end{verbatim}
\end{quote}

\item OP\_CLASS\_ATTR:设置类对象的属性或方法,需要2个操作数 \\
执行过程为:
\begin{quote}
\begin{verbatim}
r2=perm_symbols_pool[rd]
r1=dstack.pop()
r0=dstack.pop()
r0.set_attr(r2,r1)
dstack.push(r0)
\end{verbatim}
\end{quote}

\item OP\_FUNC\_OPCODE:设置函数对象的字节码,需要1个操作数,但不会弹出数据栈 \\
执行过程为:
\begin{quote}
\begin{verbatim}
r0=dstack.top()
r1=opcodes_pool[rd]
r0.set_opcode(r1)
\end{verbatim}
\end{quote}

\item OP\_LOAD\_CONST:加载常量到数据栈,需要0个操作数 \\
执行过程为:
\begin{quote}
\begin{verbatim}
acc=consts_pool[rd]
dstack.push(acc)
\end{verbatim}
\end{quote}

\item OP\_LOAD\_SYMBOL:从当前作用域中查找符号的值到数据栈,需要0个操作数 \\
执行过程为:
\begin{quote}
\begin{verbatim}
r0=symbols_pool[rd]
acc=scope.lookup(r0)
dstack.push(acc)
\end{verbatim}
\end{quote}

\item OP\_LOAD\_U\_SYMBOL:从上层作用域到顶层作用域中查找符号的值到数据栈,需要0个操作数 \\
执行过程为:
\begin{quote}
\begin{verbatim}
r0=symbols_pool[rd]
acc=scope.lookup_upper(r0)
dstack.push(acc)
\end{verbatim}
\end{quote}

\item OP\_STORE\_SYMBOL:把符号的值到存入到当前作用域,需要1个操作数 \\
执行过程为:
\begin{quote}
\begin{verbatim}
r0=dstack.pop()
r1=symbols_pool[rd]
scope.map(r1,r0)
\end{verbatim}
\end{quote}

\item OP\_STORE\_U\_SYMBOL:把符号的值到存入到上层作用域,需要1个操作数 \\
执行过程为:
\begin{quote}
\begin{verbatim}
r0=dstack.pop()
r1=symbols_pool[rd]
scope.map_upper(r1,r0)
\end{verbatim}
\end{quote}

\item OP\_IMPORT:模块导入指令,需要0个操作数 \\
执行过程为:
\begin{quote}
\begin{verbatim}
r0=symbols_pool[rd]
acc=EgThread.import(r0)
dstack.push(acc)
\end{verbatim}
\end{quote}

\item OP\_EXPORT:导出指令,需要1个操作数 \\
执行过程为:
\begin{quote}
\begin{verbatim}
r0=dstack.pop()
r1=symbols_pool[rd]
cur_module.set_attr(r1,r0)
\end{verbatim}
\end{quote}

\item OP\_BREAK:保留指令,用于辅助字节码生成,最终会被 OP\_JUMPR* 指令代替。
\item OP\_CONTINUE:保留指令,用于辅助字节码生成,最终会被 OP\_JUMPR* 指令代替。

\item OP\_JUMPR\_FORWARD:向前跳转指令,需要0个操作数 \\
执行过程为:
\begin{quote}
\begin{verbatim}
pc+=rd-3
\end{verbatim}
\end{quote}

\item OP\_JUMPR\_BACK:向后跳转指令,需要0个操作数 \\
执行过程为:
\begin{quote}
\begin{verbatim}
pc-=rd+3
\end{verbatim}
\end{quote}

\item OP\_JUMPR\_FALSE\_FORWARD:向前跳转指令,只有假时执行,需要1个操作数 \\
执行过程为:
\begin{quote}
\begin{verbatim}
r0=dstack.pop()
rs=r0.bool()
if(not rs) pc+=rd-3
\end{verbatim}
\end{quote}

\item OP\_JUMPR\_TRUE\_FORWARD:向前跳转指令,只有真时执行,需要1个操作数 \\
执行过程为:
\begin{quote}
\begin{verbatim}
r0=dstack.pop()
rs=r0.bool()
if(rs) pc+=rd-3
\end{verbatim}
\end{quote}

\item OP\_JUMPR\_FALSE\_BACK:向后跳转指令,只有假时执行,需要1个操作数 \\
执行过程为:
\begin{quote}
\begin{verbatim}
r0=dstack.pop()
rs=r0.bool()
if(not rs) pc-=rd+3
\end{verbatim}
\end{quote}

\item OP\_JUMPR\_TRUE\_BACK:向后跳转指令,只有真时执行,需要1个操作数 \\
执行过程为:
\begin{quote}
\begin{verbatim}
r0=dstack.pop()
rs=r0.bool()
if(rs) pc-=rd+3
\end{verbatim}
\end{quote}
\end{enumerate}