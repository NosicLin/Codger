\section{内存管理}
\subsection{内存管理模块的组成}
Codger的内存管理分为四部分(如下图),
\begin{quote}
\begin{verbatim}
+-------+
| Cache |                     --- level 3
+-------+---+------------+ 
| QuickMem  |     Gc     |    --- level 2
+-----------+------------+
|     MemBase            |    --- level 1
+------------------------+
|      c lib             |    --- level 0
+------------------------+
\end{verbatim}
\end{quote}
\begin{description}
\item[level 0:] C库的内存分配与释放接口malloc和free函数。
\item[level 1:] MemBase用两个作用:封装c库接口和接供页分配接口:
\begin{enumerate}
\item 封装c库内存分配与释放的接口,这样做因为c库的malloc函数在分配0字节时,不同的平台会有不同返回值。在某些平台上,malloc(0)的返回值为指向一个固定地址的指针,该指针能被free函数识别到。而其它的一些平台上,malloc(0)的返回值为NULL。不同平台上malloc(0)的不同返回值会给解析器运行时带来了不确定性。MemBase负责对c库内存接口进行封装,让分配0字节大小的内存的返回值为指向一个固定地址的指针。
\item level2 中的内存管理算法是建立在页的基础上面,要求分配的内存区域的起始地址页对齐,并且大小为页的大小。 C库的malloc函数只能分配指定大小字节的内存区域,不能指定区域的起始地址为页对齐,在C库中也没有提供以页为单位的分配接口。所以需要MemBase实现页分配与释放接口AllocPage和FreePage。
\end{enumerate}
\item[levle 2:] 由QuickMem和Gc两部分构成:
\begin{enumerate}
\item QuickMem和c库一样提供内存分配与释放接口,QuickMem采用以页为基础的内存管理算法,在小内存\footnote{小内存:大约小于300字节}分配时,内存分配的速度比c库能快到3\verb|~|4倍。从QuickMem中分配出去的内存必须调用释放函数归还,否则会造成内存遗漏。
\item Gc的全称的Garbage Collection,Codger的所有对象比须从Gc中分配,这些对象不需要调用释放函数归还给Gc,Gc也没提供对象释放接口。每隔一段时间,Gc的垃圾回收例程会工作,收集那些在程序中不能被访问到的对象,这些对象被称作为垃圾对象。垃圾对象被回收后,它所占用的内存可被重新利用,Gc的垃圾回收算法为代式节点复制法。
\end{enumerate}
\item[level 3:] 虚拟机在运行时,会不停的分配也释放虚拟机本身管理所需要的内存,分配与释放频率很高的内存块会被缓存一定的数量。例如每当一个Codger方法分调用时,虚拟机会创建一个栈帧,当Codger方法返回时,栈帧会被释放,如果分次栈帧创建都从QuickMem中分配,释放时在返回给QuickMem,这样一来一回,会耗掉很多时间。
\end{description}

\subsection{页分配算法}
页是整个内存管理模块的核心,QuickMem模块基于页实现快速内存分配,Gc模块中的堆也是基于页实现的。MemBase模块中实现的页分配接口,以供level2使用。它的实现方法为每次利用c库的malloc函数每配一大块内存,这一大块内存被称作为域(Area)。在Codger中域的大小为1M,从域的起始地起开始,找到第一个页对齐的起地址,从该地址开始,以页大小为基础分配内存,直到最后的内存小于页的大小。Area最前面和最尾部的一小段内存都无法使用。

MemBase模块中,用AreaHeader来表示每一个Area,如果Area中的所有页都被会配完,则Area会被放到full\_list中。如果Area有页可以分配,则它会被放到free\_list中, cur\_area 表示用于当前分配页的Area。 \\
MemBase的PageAlloc算法为:
\begin{quote}
\begin{verbatim}
MemBase.AllocPage()
    if cur_area==Nil
        if free_list.not_empty()      #如果free_list 不分空
             cur_area=free_list.take_front() #取出free_list中第一个Area
        else
             cur_area=AllocNewArea()  #分配一个新的Area
        end
     end 
     page=cur_area.alloc()       #从当前area中分出一页
     if cur_area.full()          #如果area没有空闲页,则放到full_list中
        full_list.add(cur_area)
        cur_area=Nil
     end
     return page
end 
\end{verbatim}
\end{quote}
MemBase的PageFree算法:
\begin{quote}
\begin{verbatim}
MemBase.FreePage(page)
    area = FindPageArea(page)    #找到page所在的area
    if area.full()       #如果area以前没有空闲页,则它在full_list中
       full_list.del(area)     #现在一页被释放后,从full_list中删除
       free_list.add_front(are)  #放入到free_list中
    end
    area.FreePage(page)          #把页page还给area
    if area.alloced_page_nu==0   #如果area的全部页都被归还
       if area!=cur_area        
           free_list.del(area)   #则把内存还给操作系统
           FreeArea(area)
       end
    end
end 
\end{verbatim}
\end{quote}
其中FindPageArea首先判断page是否在cur\_area中,判断方法是通过page的地起是否在area的地起范围内,然后再遍历free\_list中的每一个area,最后是full\_list。

AreaHeader用来管理一个Area中的所有数据,包括Area中的空闲页,总共可分配的页的数量,已分配页的数量等,它的结构如下:
\begin{quote}
\begin{verbatim}
class AreaHeader 
    attr area_link         #用于full_list或者是free_list
    attr max_page_nu       #总共可分配的页的数量
    attr alloced_page_nu   #已分配页的数量
    attr free_pos          #指向下一页的地址
    attr free_page_list    #空闲页链表
end 
\end{verbatim}
\end{quote}
当创建一个AreaHeader需要计算出它可分配页的最大数量,free\_pos从域的起始地起开始,指向第一个页对齐的起地址。其它属性则被初始化空或者0。可分配的最大数量的计算公式为:
\[\frac{AreaSize}{PageSize}-1\]
从AreaHeader中每配一页的算法为:
\begin{quote}
\begin{verbatim}
AreaHeader.AllocPage()
    if free_page_list.not_empty()   #如果free_page_list不为空
        page = free_page_list.take_front()  #则从free_page_list中取出
    else
        page=free_pos               #在Area中划分出一页
        free_pos+=PageSize          #更新free_pos的位置
    end 
    alloced_page_nu+=1              #更新已分配页的数量
    return page                     #返回页
end 
\end{verbatim}
\end{quote}
MemBase的AllocPage算法保证,不向没有空闲页的area中分配页。 把页归还 AreaHeader 算法为:
\begin{quote}
\begin{verbatim}
AreaHeader.FreePage(page)
    free_page_list.add_front(page)
    alloced_page_nu-=1
end 
\end{verbatim}
\end{quote}

\subsection{内存分配算法}
c库的内存分配效率已经非常优秀,但毕竟c库的内存分配算法是一种通用算法,它能适应不同大小的内存分配请求,正是因为这种通用性,使内存中很容易堆积碎片。当内存碎片过多时,分配效率也会随着下降。QuickMem这一个模块是针对小内存分配设计的,小内存的界线SmallSize,可以在修改源码中的宏定义的值来改变,在Codger中小内存的定义为256字节。

QuickMem工作原理是把不同大小的内存进行分组,每一组分配内存的单位是8个倍数。当需要分配的内存大小为1\verb|~|8时,从每0组中分配,分配内存大小为8字节;当分配的内存大小为9\verb|~|16时,从每1组中分配,分配内存大小为16字节;依此类推下去。如果需要分配的内存大于SmallSize,则QuickMem调用C库的内存分配函数。QuickMem的这种工作方式,使得从QuickMem分配到的内存,比需要的内存多出0\verb|~|7个字节,造成一定数量的内存浪费,但这种浪费并不是很多。现在的cpu大多数都是32位,编译器在编译程序时,会强制4字节对齐以获得更好的Cache性能,即使结构体只有一个char的成员,该结构体也会占用4字节。所以从QuickMem中每配到的内存只有两种情况:1)比需要的大出4字节;2)刚好等于需要的内存。

QuickMem内存分配以组为单位,当需要每配一个小于 SmallSize 大小的内存size时,首先需要计算出它属于那一组,因为分配单位是8个倍数,只需要将size右移3位,就能得到它所在的分组,然后出它从所在分组中分配内存。

假设现在的分组为N,分组调用MemBase页分配接口,用于获得一页,然后它在这一页的开头写入一些管理数据称为PageHeader(大小保证8字节对齐)。最后把页剩余的空间划分成该分组每配单位((N+1)*8)的相等的多个空间,用于分配。分组的数据结构与前面绍介的MemBase的管理数据相同,有free\_list, full\_list,cur\_page。free\_list保存所有还能分配空间的页,full\_list中的页没有剩余空间可以分配,cur\_page表示用于分配的当前页。PageHeader与AreaHeader的数据结构相似,只不过分配单位的大小不同,AreaHeader 一次性分配一页,PageHeader一次性分配(N+1)*8字节大小,它们的分配算法原理相同。

当程序释放内存给QuickMem时,释放的内存有两种情况,每一种情况为释放的内存属于小内存,来自于分组中,每二种情况为释放的来自于C库,QuickMem需要对它们进行判断。这里首先讨论当内存属于小内存时的情况。每个小内存都是从分组中每配,分组在每一个页的开头(PageHeader)都写入管理数据(如下图),只要释放内存的地址的低位屏蔽掉就可以获得它所在的PageHeader。假设释放内存的地址为addr,计算公式为: \verb|addr&(~(PageSize-1))|。得到释放内存所在的PageHeader,然后调用PageHeader.Free(addr)
\begin{quote}
\begin{verbatim}
PageBeginAddr           addr           PageEndAddr
   |                      |                |
   v                      V                V
   +----------+--------------\\------------+
   |PageHeader|    ...        \\    ....   |
   +----------+----------------\\----------+
\end{verbatim}
\end{quote}
每二种情况是释放的内存来自于C库,QuickMem分在每个分配的页一个唯一的标识符,该标识符保存在PageHeader中,在QuickMem中有一个一维数组IDX,如果某一页的标识符为n,那么IDX[n]中就保存就该页的地址(如下图):
\begin{quote}
\begin{verbatim}
        _____________\\ ___________
IDX->  |___|___|___|__\\___|___|___|
        /    \         \\       |
       V      V                 V
    +-----+  +-----+          +-----+
    |     |  |     |          |     |
    |     |  |     |          |     |
    +-----+  +-----+          +-----+
\end{verbatim}
\end{quote}
判断释放内存的来源时,首先能过地址得到PageHeader,该PageHeader不一定是一个有效的数据,然后出PageHeader中取出标识符n,有下面几种情况:
\begin{enumerate}
\item n的值不在IDX的范围内,则可判定它来自于C库
\item n的值在标识符范围内时:
\begin{enumerate}
\item IDX[n]!=PageHeader,则表明需要释放的内存来自于C库
\item IDX[n]==PageHeader,则表明需要释放的内存来自于分组
\end{enumerate}
\end{enumerate}
因为操作系统内部采用虚拟分页技术,在掉取PageHeader标识值时,即使释放内存来自于C库,也不会发生缺页异常或者是内存错误,而且只是提取,没有赋值,原来的值不会有任何的更改,这是一个安全的过程。

\subsection{垃圾回收算法}
Codger中的所有对象内存都必须从GC模块中分配,对象分配后,不必在程序中显示的释放,Gc模块也没有提供释放接口,如果一个对象不能直接或者间接的到达,那这个对象则被称为垃圾对象。垃圾对象会一直存在,直到垃圾回收例程启动,将它们收集,并释放它们所占用的内存。
\subsubsection{堆的构成}
在Gc模块中总共有四种不同类型的堆:静态区,惰性区,年老区,年轻区,不同类型的堆有不同的用途。
\begin{enumerate}
\item 静态区(Static):用于分配静态数据,其它类型的堆的对象不会被提升到静态区。从静态区中分配的对象不会被回收,即使对象已经死亡,所以需要确保对象在解析器整个运行的生命周期类,不会成为垃圾对象。它被用于以下两种情况:
\begin{enumerate}
\item 解析器中某些非常固定的对象,它们的生命周期与解析器的生命周期相同。例如Codger中布尔值的false对象与true对象,Nil对象等都是从静态数据区中分配。
\item 因为Codger中垃圾回收算法采用的是分代式节点复法,垃圾回收例程可能会把对象复制到另一块区域,对象被复制后地址也被改变了。然而某些扩展模块中,它们可能要求某些对象在其生命周期内,它的地址不能被改变。在静态区分配的对象,这点能够得到保证。
\end{enumerate}
\item 惰性区(Inertia):用于分配程序中极少会变化对象,这些对象不能确保在解析器的整个生命周期内不会死亡,但是在大多数情况下,它们会存活很长一段时间或者是存活到解析器退出。惰性区中的对象会被回收,它回收频率很低,并且类型的堆对象不会被提升到惰性区。例如,解析器在解析源程序时,源程序中出现的常量,符号(字符串对象),模块对象,类对象等都会从隋性区分配。
\item 年老区(Old):用于接受年轻区被提升上来的对象,年老区的回收频率相对比年轻区的低的很多。
\item 年轻区(Young):虚拟机运行时,所有的对象都是从年轻区分配,Codger是一门动态语言,每个时刻都会有大量的对象会被创建,但这些对象大多数都是计算过程中产生的临时对象,存活率很低,通常情况下是低于10\%,让这些对象在年轻区分配,垃圾回收例程在多数情况下回收年轻区的垃圾对象,这样就只需要把低于10\%的对象复制到另一个半区,很大程度上提高垃圾回收例程的效率。如果对象在年轻区被回收两次后依然存活,它就会被提升到年老区。
\end{enumerate}

Gc模块的堆和QuickMem一样是基于页实现了,它并不是一块连续的区域,而是由多个页连在一块的(如图\ref{fig:gc_heap},如果需要分配一个对象,则从当前页中划分出一块区域分给对象,如果当前页并不会划分出中够大小的区域,那么堆会向MemBase申请一块新的页,把新页放入到page\_list中,并置cur\_page为新分配的页。
\begin{figure}
\begin{verbatim}
 page_list                                  cur_page
     \                                         |
      V                                        V
    +------+ ---> +------+ --->  .... -->+------+      
    |      |      |      |               |      |
    |      |      |      |       ....    |      |
    |      |      |      |               |      |
    +------+      +------+               +------+
\end{verbatim}
\caption{Gc堆的构成}
\label{fig:gc_heap}
\end{figure}
堆的分配内存描述为:
\begin{quote}
\begin{verbatim}
Heap.Alloc(size)
    if !cur_page.CanAlloc(size)   #如果当前页不能分配size大小的内存
       page=MemBase.AllocPage()    #申请一块新的页
       HeapInitPage(page)          #初始化页数据
       page_list.add_tail(page)    #把新页放入到page_list尾部
       cur_page=page               #更新cur_page
    end
    return cur_page.Alloc(size)
end
\end{verbatim}
\end{quote}
Codger最大的对象也没有超过100字节,分配出的新页肯定能成功的划分出一块内存区域。和QuickMem中的页一样,Gc模块的页也会在页开始的区域写入一些管理数据,称作为GcPageHeader。因为Codger采用节点复制法,节点复制法一个优势在于它有很快的内存分配速度,几乎可以和基于栈式内存分配相媲美。在内存分配时,只需要每次向下划分一块区域即可。在垃圾回收例程工作时,会把一个半区的成活的对象复制到另一个半区,原来半区中的所有页都会被释放。下一次内存分配会从另一个新半区开始分配,整个过程不会存在内存碎片。GcPageHeader中有两上数据,一个保存Page有空闲空间的位置free\_pos,另一个保存Page的最大可分配位置max\_pos。在Page中分配内存描述为:
\begin{quote}
\begin{verbatim}
GcPageHeader.Alloc(size)
    ptr=free_pos
    free_pos+=round8(size)   #保证8字节对齐
    return ptr
end 
\end{verbatim}
\end{quote}
从上面代码可以看出,分配只需要向下划分一块区域即可,判断页是否可分配的代码为:
\begin{quote}
\begin{verbatim}
GcPageHeader.CanAlloc(size)
    if free_pos+size>max_pos
       return false
    else 
       return true
    end
end 
\end{verbatim}
\end{quote}

\subsubsection{回收例程}
垃圾回收例程的算法使用的是Cheney的节点复制收集法\footnote{详见:《垃圾收集》 Richard Jones,Rafael Lins著. 谢之易 译. 第6.1节},每次当垃圾回收例程工作时,都会有确定回收的级别CollectionLevel,每个不同类型的堆区的回收级别关系为:\verb|静态区>惰性区>年老区>年轻区|。CollectionLevel确定后,和CollectionLevel相等或小于的堆都要参与回收。例如,CollectionLevel为年老区时,则年老区和年轻区都会被回收,而惰性区和静态区则不会被回收。CollectionLevel的最大级别为惰性区,所以静态区永远都不会被回收。参与回收的堆,都会把存活的对象从当前半区移到与之相应的另一个半区。如果对象处于年轻半区,并且在上一次回收时存活了下来,则它们被提升到老年分代。如果在对象在上次回收之后创建,并且这次回收存活了下来,则会标记对象在下次回收时提升它。

对于回收级别大于CollectionLevel的堆,它们同样也会被更新。Codger现在使用一种方法类似于线性扫描,线性扫描算法为依次扫描堆中所有对象,看该对象是引用了低级别的对象,如果是,则更新该对象。在每个对象中,如果引用了一个对象引用了底级别的对象,那么该对象的低级别引用位会被置1。这一位可以用于判断是否引用了底级别的对象。
Codger在实现不仅在对象中保存标志位,同样该对象所在的页的头部(GrPageHeader)也会保存该标志位,这样就可以页头部来判断该页中是否有对象引用了底级别的对象,如果则扫描该页中的所有对象,找到所有低级别引用位为被置位的对象,并更新该对象,如果没有则跳过该页。

\subsubsection{拦截器}
如果一个对象引用了另一个对象,拦截器会对这个过程进行拦截,判断其是否引用的是低级别的对象,如果是则置对象的低级别引用位为1,同时更新该对象所在页头的标记位。在Codger中,为了提高内存利用率,在每个对象中并不会保存它所在堆的级别,它所在堆的级别信息保存在对象所在的页头中,每个对象获取页头的方法与QuickMem中的方法一样,假设对象的地址为addr,则可以通过\verb|addr&(~(PageSize-1))|得到。拦载器的工作过程为:
\begin{quote}
\begin{verbatim}
Intercept(x,y)
    header_x=x&(~(PageSize-1))         #得到对象x所在的页头
    header_y=y&(~(PageSize-1))         #得到对象y所在的页头
    if header_x.level>header_y.level   #x所在的级别大于y所在的级别
        x.rel_low=1                    #标记对象x低分代引用位
        header_x.ref_low=1             #标记x所在页头代分代引用位
    end
end
\end{verbatim}
\end{quote}

\subsubsection{标记位}
Codger中的对象总共需要2个标记位,一个为标记该对象是否需要在下次垃圾回收时提升到年老分代,一个用于标记对象是否引用了低级别的对象。这两个位如果单独给它们分配一个long型域用于保存它们,显得太浪费,这会让每个对象占用的空间多上4字节。即使在定义时,使用char型,编译器在编译时,会考虑在32位平台上4字节对齐,最后每个对象还是会多占用4字节的空间。

在Codger中,每个对象都有一个指针域用于指向该对象类型的信息,如果类型信息是从QuickMem中动态分配,QuickMem能保证得到的指针是8字节对齐,如果类型信息是在程序中使用静态数据,也可通过编译器的扩展语法保证所得到的地址为4字节对齐。保证对象类型信息至少4字节对齐后,它的低2位就可用于对象的标记位,当需要获得对象类型信息时,只需要将低两位屏蔽掉即可。Codger在实现的代码如下,该结构体只适用于CPU为小端字节:
\begin{quote}
\begin{verbatim} 
struct  gr_object 
{
    union{
        struct gr_type_info* g_type;
        struct 
        {
            long g_ref_low:1;  
            long g_upgrade:1;
            long g_reverse:30;
        };
    };
};
\end{verbatim}
\end{quote}

















